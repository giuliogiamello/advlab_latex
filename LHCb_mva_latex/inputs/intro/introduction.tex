\chapter*{Introduction}
\addcontentsline{toc}{chapter}{Introduction} % useful to add index in Table of contents

Particle physics is at a point of a very statistical and gradual unravelling of the cosmic mystery that is our reality. Most of the processes that are of interest in our discipline occur within time and length scales that are practically impossible to record or observe. The strong and electroweak interactions that we are trying to understand can only be probed by the rigorous analysis of the final and initial states of the particles involved in these interactions. The strategy scientists decided to employ in order to study these fundamental interactions has evolved ever so slightly since the 50's, when CERN was founded. The true evolution occurred in our detector and computational capacity. Today, in Large Hadron Collider (LHC), we are colliding protons with 13 TeV center-of-mass energies at \( 40 \times 10^{16} \) times per second. Such high energies and high rate of collisions unlock the possibility of analyzing rarer processes or those that are suppressed at lower energies. This increased capacity comes with a caviat; the perfect physics breeding ground we have at our collision point results in huge amounts of physical processes to occur per collision and accordingly, massive amount of data is registered for each collision event.\\

Our purpose built detectors manage to record the aftermath or the final states of the fundamental particle interactions very well; however, the flipside of the coin, the massive data that we collect needs to be analyzed with many layers of similarly purposefully designed hardware systems, software systems and algorithms. Usually, we are interested in figuring out the details of a specific process. The detector measurement for any process in LHC is buried under every other process that happens at the same event, therefore we need to come up with a strategy to select only the signal for the process we want to study. In this report, we are going to use the aid of the relatively high technological advance of computers to build a machine learning algorithm (Multi-Variate Classifier) that will be able extract the signal for the process we are looking for from the dataset we have from one of the four detectors at the LHC, the LHC-beauty detector.