\chapter*{Conclusions}\label{chap:conclusion}
\addcontentsline{toc}{chapter}{Conclusions}

In this laboratory course, we investigated the rare decay $B^{0}_{s} \xrightarrow{} \psi(2S)K^{0}_{S}$, which is experimentally challenging due to its low yield in the presence of significant combinatorial background, including a nearby peak from the more abundant $B^{0}_d \xrightarrow{} \psi(2S)K^{0}_{S}$ decay which we used as a control channel. To conduct the search, we employed a multivariate classifier based on a Boosted Decision Tree (BDT) that exploits the discriminating power of multiple kinematic and topological variables. We trained our classifier using select variables specifically chosen to be unbiased, well simulated and well discriminating between signal and background. The classifier effectively suppressed combinatorial background while retaining our signal. This approach enabled a clearer separation between the $B^{0}_{s}$ and $B^{0}_{d}$ components in the invariant mass distribution, massively improving the signal-to-background ratio. The use of machine learning proved to be vital in isolating this suppressed decay due to the combinatorial background. Our findings of $5.68 \hspace{0.1em} \sigma$ suggests that there is a good chance that what is observed is the $B^{0}_{s}$ particle and not to a statistical fluctuation.
