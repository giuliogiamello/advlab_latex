\chapter*{Conclusions}\label{chap:conclusion}
\addcontentsline{toc}{chapter}{Conclusions}

In this lab, we investigated the rare decay $B^{0}_{s} \xrightarrow{} \psi(2S)K^{0}_{S}$, which is experimentally challenging due to its low yield in the presence of significant background, including a nearby peak from the more abundant $B^{0}_d \xrightarrow{} \psi(2S)K^{0}_{S}$ decay. To conduct the search, we employed a multivariate classifier based on a Boosted Decision Tree (BDT) that exploits the discriminating power of multiple kinematic and topological variables. We trained our classifier using select variables (also called features) specifically chosen to be unbiased, well simulated and well discriminating. The classifier effectively suppressed combinatorial and partially reconstructed background while retaining our signal. This approach enabled a clearer separation between the $B^{0}_{s}$ and $B^{0}_{d}$ components in the invariant mass distribution, massively improving the signal-to-background ratio. The use of machine learning proved crucial in isolating this suppressed decay. This result suggests that there is a good chance that the observed $B^{0}_{s}$ particle and not to a statistical fluctuation.
