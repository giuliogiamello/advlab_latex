\chapter*{Introduction}
\addcontentsline{toc}{chapter}{Introduction}  % useful to add index in Table of contents 

The origin and evolution of our universe is an open question still nowadays. One of the most profound mysteries is the matter-antimatter asymmetry. From what we know, the Big Bang should have produced equal amounts of matter and antimatter; however, our universe is overwhelmingly composed of matter. This imbalance poses a grand challenge to our understanding of fundamental physics and the Standard Model of particle physics; as there is currently no known mechanism that can fully account for the conditions that lead to a matter-dominated universe. For this lab, we are going to study one of the possible explanations for the matter-antimatter asymmetry in the form of decay rate comparisons for matter and antimatter states.

Most of the processes that are of interest in our discipline occur within time and length scales that are practically impossible to record or observe. The strong and electroweak interactions that we are trying to understand can only be probed by the rigorous analysis of the final and initial states of the particles involved in these interactions. The strategy scientists decided to employ in order to study these fundamental interactions has evolved ever so slightly since the 50's, when CERN was founded. The main goal of the strategies being to minimize the interference from all the technical difficulties as much as possible and extract the useful science from the experimental data. Today, in Large Hadron Collider (LHC), we are colliding protons with 13 TeV center-of-mass energies at \( 40 \times 10^{16} \) times per second. Such high energies and high rate of collisions unlock the possibility of analyzing many processes. Our purpose built detectors manage to record the aftermath or the final states of the fundamental particle interactions very well; however, the flipside of the coin, the massive data that we collect needs to be analyzed with many layers of similarly purposefully designed hardware systems, software systems and reconstruction. The detector measurement for any process in LHC is buried under every other process that happens at the same time. Since we are interested in figuring out the details of a specific process, we always need to come up with a strategy to select only the signal for the process we want to study, which is what we will be following up fully in this lab.

In more detail: the decay process of our interest in this analysis is the B-meson decay $B^+ \rightarrow K^+ K^- K^+$.
