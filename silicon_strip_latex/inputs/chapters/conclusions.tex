\chapter*{Conclusions}\label{chap:conclusion}
\addcontentsline{toc}{chapter}{Conclusions}

This experimental investigation focused on characterizing the operational properties of a silicon radiation detector using both optical excitation (laser) and a radioactive ($^{90}\mathrm{Sr}$) source. Initial calibration procedures preceded the determination of geometric parameters for the detection system. The strip separation was evaluated as 40 $\mu$m, while the laser beam diameter was measured at 30 $\mu$m.

Comparative analysis of detector responses revealed distinct signal characteristics between excitation sources: $\beta$ decay events from the radioactive source produced lower-amplitude signals compared to laser-induced excitations, attributable to the electron emission in radioactive decay processes. The mean laser penetration depth in silicon was quantified as $a = (252 \pm 72)  \mu\text{m}$.

Subsequent signal processing involved converting pulse-height distributions from ADC units to energy values in keV. The resulting energy distribution conformed to a Landau profile, with a most probable value of 20.20 keV and a mean energy of 17.97 keV.
%----------------------------------------------------%
